\documentclass[11pt,a4paper]{article}

% Balíčky pre slovenčinu a formátovanie
\usepackage[utf8]{inputenc}
\usepackage[T1]{fontenc}
\usepackage[slovak]{babel}
\usepackage{graphicx}      % Pre vkladanie obrázkov
\usepackage{geometry}      % Okraje stránky
\usepackage{hyperref}      % Klikateľný obsah
\usepackage{tabularx}      % Tabuľky
\usepackage{indentfirst}   % Odsadenie prvého odseku
\usepackage{amsmath}       % Matematika
\usepackage{listings}      % Pre kód a gramatiku

% Nastavenie okrajov (štandard pre VUT FIT)
\geometry{
    textheight=230mm,
    textwidth=160mm,
    top=30mm,
    left=25mm,
    right=25mm
}

\begin{document}

% ---------------- TITULNÁ STRANA ----------------
\begin{titlepage}
    \begin{center}
        \includegraphics[width=0.6\textwidth]{logo_fit.png} % Tu treba nahrať logo FIT, ak nemáte, riadok zakomentujte (%)
        \vspace{3cm}

        {\Large \textsc{Vysoké učení technické v Brně}}\\
        {\Large \textsc{Fakulta informačních technologií}}\\
        \vspace{2cm}

        {\Huge \textbf{Dokumentácia k projektu IFJ/IAL}}\\
        \vspace{1cm}
        {\Large Implementácia prekladača imperatívneho jazyka IFJ25}\\
        \vspace{3cm}

        {\large \textbf{Tím xmicham00, varianta vv-BVS}}
        \vspace{0.3cm}
        
        {\large \textbf{Implementované rozšírenia: žiadne}}
        \vspace{1cm}

        \begin{table}[h!]
            \centering
            \begin{tabular}{l l l l}
                \textbf{Martin Michálik} & (xmicham00) & 25\% & [Vedúci] \\
                \textbf{Matúš Magyar}    & (xmagyam00) & 25\% & \\
                \textbf{Šimon Škoda}     & (xskodas00) & 25\% & \\
                \textbf{Jaroslav Vrbiniak} & (xvrbinj00) & 25\% & \\
            \end{tabular}
        \end{table}

        \vfill
        Brno, \today
    \end{center}
\end{titlepage}

% ---------------- OBSAH ----------------
\tableofcontents
\newpage

% ---------------- TEXT DOKUMENTU ----------------

\section{Úvod}
Táto dokumentácia popisuje návrh a implementáciu prekladača jazyka IFJ25 do medzikódu IFJcode25. Cieľom projektu bolo vytvoriť funkčný prekladač, ktorý spracúva zdrojový kód v jazyku IFJ25 a generuje ekvivalentný program v jazyku IFJcode25.

Prekladač sa skladá z niekoľkých na seba nadväzujúcich častí: lexikálny analyzátor (scanner), syntaktický analyzátor (parser) využívajúci metódu rekurzívneho zostupu, precedenčný syntaktický analyzátor pre výrazy, sémantický analyzátor a generátor cieľového kódu.

Projekt bol implementovaný v jazyku C a vyvíjaný v termíne od 14.10.2025 do 03.12.2025.

\section{Architektúra projektu}
Zdrojové súbory v projekte sú rozdelené podľa funkcie:

\begin{itemize}
    \item \texttt{ast.c/h} -- abstraktný syntaktický strom
    \item \texttt{builtins.c/h} -- spracovanie vstavaných funkcií
    \item \texttt{errors.c/h} -- správa chybových kódov
    \item \texttt{expression.c/h} -- precedenčná analýza výrazov
    \item \texttt{generator.c/h} -- generovanie IFJcode25
    \item \texttt{main.c} -- vstupný bod programu
    \item \texttt{parser.c/h} -- syntaktická analýza a riadenie prekladu
    \item \texttt{scanner.c/h} -- lexikálna analýza
    \item \texttt{symtable.c/h} -- tabuľka symbolov (AVL strom)
\end{itemize}

\section{Lexikálna analýza}
Modul \texttt{scanner.c} implementuje lexikálny analyzátor ako deterministický konečný automat (DKA). Hlavná funkcia \texttt{get\_next\_token()} používa veľký \texttt{switch} na vetvenie podľa aktuálneho znaku, čo zodpovedá stavom DKA.

\subsection{Kľúčové implementácie}
Pri spracovaní reťazcových literálov používame dynamický buffer s počiatočnou kapacitou 256 bajtov, ktorý sa automaticky zdvojnásobuje pri preplnení. Táto technika zabezpečuje efektívne spracovanie aj veľmi dlhých reťazcov bez plytvenia pamäťou.

Multi-line stringy sú rozpoznávané hľadaním trojice \texttt{"""} -- scanner najprv rozpozná bežný string, a ak začína ďalším \texttt{"}, prechádza do režimu viacriadkového reťazca. V tomto režime sa znaky ukladajú priamo do bufferu až kým sa nenájde ukončovacia trojica.

Vnorené blokové komentáre sú implementované rekurzívne pomocou funkcie \texttt{ignore\_multiline\_comment()}. Pri nájdení \texttt{/*} vo vnútri komentára sa funkcia zavolá rekurzívne, čo automaticky spracuje ľubovoľne hlboké vnorenie.

Hexadecimálne čísla rozpoznávame po prefixi \texttt{0x} alebo \texttt{0X} pomocou funkcie \texttt{strtoull()} so základom 16. Exponenciálny zápis je rozpoznaný pri výskyte znaku \texttt{e}/\texttt{E} a výsledok sa vypočítava pomocou \texttt{powf()}/\texttt{pow()}.

\clearpage

\subsection{Diagram konečného automatu}
\begin{figure}[h!]
    \centering
    \includegraphics[width=\textwidth]{fsm.png} 
    \caption{Diagram konečného automatu lexikálneho analyzátora. Zelené uzly s dvojitým rámom predstavujú finálne akceptujúce stavy (tokeny), žlté uzly sú prechodové stavy, sivé uzly predstavujú komentáre (ktoré sa ignorujú a negenerujú tokeny), modrý uzol s hrubým rámom je počiatočný stav. Notácia ¬X znamená \uv{všetky znaky okrem X}.}
    \label{fig:fsm}
\end{figure}

\noindent\textbf{Rozpoznávané tokeny:} 
\begin{itemize}
    \item \textbf{Operátory:} aritmetické (\texttt{+}, \texttt{-}, \texttt{*}, \texttt{/}), priradenie (\texttt{=}), porovnávacie (\texttt{==}, \texttt{!=}, \texttt{<}, \texttt{<=}, \texttt{>}, \texttt{>=})
    \item \textbf{Oddeľovače:} zátvorky (\texttt{(}, \texttt{)}, \texttt{\{}, \texttt{\}}), čiarka, bodka, nový riadok
    \item \textbf{Číselné literály:} celé čísla, desatinné, hexadecimálne (prefix \texttt{0x}), exponenciálne (s~\texttt{e}/\texttt{E})
    \item \textbf{Reťazce:} jednoriadkové (\texttt{"text"}), viacriadkové (\texttt{"""text"""}), escape sekvencie (\texttt{\textbackslash n}, \texttt{\textbackslash t}, \texttt{\textbackslash r}, \texttt{\textbackslash\textbackslash}, \texttt{\textbackslash"}, \texttt{\textbackslash 0})
    \item \textbf{Identifikátory:} bežné identifikátory a kľúčové slová (\texttt{class}, \texttt{static}, \texttt{var}, \texttt{if}, \texttt{else}, \texttt{while}, \texttt{return}, \texttt{null}, \texttt{Num}, \texttt{String}, \texttt{Null}, \texttt{import}, \texttt{for}, \texttt{is}, \texttt{Ifj})
    \item \textbf{Globálne premenné:} identifikátory s prefixom \texttt{\_\_} (napr. \texttt{\_\_counter})
    \item \textbf{Komentáre:} jednoriadkové (\texttt{//}), viacriadkové vnorené (\texttt{/* */})
\end{itemize}

\section{Syntaktická analýza}
Parser je implementovaný v module \texttt{parser.c} metódou rekurzívneho zostupu.

\subsection{Správa tokenov}
Parser obsahuje mechanizmus \texttt{putback\_token()}, ktorý umožňuje vrátiť jeden token späť do prúdu. Toto je nevyhnutné v situáciách, keď potrebujeme \uv{nahliadnuť dopredu} na nasledujúci token bez jeho konzumovania.

\subsection{Name mangling}
Pre podporu preťažovania funkcií (rovnaké meno, rôzny počet parametrov) používame \emph{name mangling}. Funkcia \texttt{foo} s dvoma parametrami sa interne ukladá ako \texttt{foo\$arity2}. Podobne gettery a settery používajú sufixy \texttt{\$get} a \texttt{\$set}. Tento prístup umožňuje jednoznačnú identifikáciu každej varianty funkcie v tabuľke symbolov.

\subsection{Odložená validácia volaní}
Volania funkcií sa zaznamenávajú do linked listu \texttt{function\_calls\_list} obsahujúceho názov, aritu a číslo riadku. Samotná validácia prebieha až po spracovaní celého programu funkciou \texttt{validate\_function\_calls()}. To umožňuje volať funkcie, ktoré sú definované až neskôr v kóde.

\subsection{Spracovanie výrazov}
Modul \texttt{expression.c} implementuje precedenčnú analýzu pomocou zásobníkovej metódy. Precedenčná tabuľka 15x15 definuje vzťahy medzi operátormi pomocou symbolov \texttt{<} (posun), \texttt{>} (redukcia), \texttt{=} (rovnosť) a prázdne pole (chyba). Analýza konštruuje abstraktný syntaktický strom (AST), ktorý sa následne použije pre generovanie kódu.

\subsubsection{Precedenčná tabuľka}
\begin{table}[h!]
\centering
\scriptsize
\begin{tabular}{|c|c|c|c|c|c|c|c|c|c|c|c|c|c|c|c|}
\hline
& \textbf{(} & \textbf{)} & \textbf{i} & \textbf{+} & \textbf{-} & \textbf{*} & \textbf{/} & \textbf{<} & \textbf{<=} & \textbf{>} & \textbf{>=} & \textbf{==} & \textbf{!=} & \textbf{is} & \textbf{\$} \\
\hline
\textbf{(}  & < & = & < & < & < & < & < & < & < & < & < & < & < & < &   \\
\textbf{)}  &   & > &   & > & > & > & > & > & > & > & > & > & > & > & > \\
\textbf{i}  &   & > &   & > & > & > & > & > & > & > & > & > & > & > & > \\
\textbf{+}  & < & > & < & > & > & < & < & > & > & > & > & > & > & > & > \\
\textbf{-}  & < & > & < & > & > & < & < & > & > & > & > & > & > & > & > \\
\textbf{*}  & < & > & < & > & > & > & > & > & > & > & > & > & > & > & > \\
\textbf{/}  & < & > & < & > & > & > & > & > & > & > & > & > & > & > & > \\
\textbf{<}  & < & > & < & < & < & < & < &   &   &   &   &   &   &   & > \\
\textbf{<=} & < & > & < & < & < & < & < &   &   &   &   &   &   &   & > \\
\textbf{>}  & < & > & < & < & < & < & < &   &   &   &   &   &   &   & > \\
\textbf{>=} & < & > & < & < & < & < & < &   &   &   &   &   &   &   & > \\
\textbf{==} & < & > & < & < & < & < & < &   &   &   &   &   &   &   & > \\
\textbf{!=} & < & > & < & < & < & < & < &   &   &   &   &   &   &   & > \\
\textbf{is} & < & > & < & < & < & < & < &   &   &   &   &   &   &   & > \\
\textbf{\$} & < &   & < & < & < & < & < & < & < & < & < & < & < & < &   \\
\hline
\end{tabular}
\caption{Precedenčná tabuľka pre výrazy (i = identifikátor/literál, \$ = koniec výrazu)}
\end{table}

Priamo počas redukcie sa vykonáva typová kontrola literálov -- funkcia \texttt{check\_binary\_literal\_types()} overuje kompatibilitu typov pre binárne operácie.

\subsection{LL gramatika}
Pre syntaktickú analýzu jazykových konštrukcií bola navrhnutá LL(1) gramatika. Gramatika obsahuje 24 neterminálov a 56 pravidiel:

\begin{verbatim}
1.  <prolog>         -> import STRING for Ifj EOL <eols>
2.  <program>        -> class IDENTIFIER { EOL <func_list> }
3.  <main_func>      -> static IDENTIFIER ( ) <block>
4.  <func_list>      -> }
5.  <func_list>      -> <func> <func_list>
6.  <func>           -> static IDENTIFIER ( <param_list> ) <block>
7.  <func>           -> static IDENTIFIER { <block>
8.  <func>           -> static IDENTIFIER = ( IDENTIFIER ) <block>
9.  <param_list>     -> )
10. <param_list>     -> IDENTIFIER <param_list_tail>
11. <param_list_tail>-> )
12. <param_list_tail>-> , IDENTIFIER <param_list_tail>
13. <block>          -> { EOL <stmt_list> }
14. <stmt_list>      -> }
15. <stmt_list>      -> <stmt> EOL <stmt_list>
16. <stmt>           -> <var_decl>
17. <stmt>           -> <assign>
18. <stmt>           -> <if_stmt>
19. <stmt>           -> <while_stmt>
20. <stmt>           -> <return_stmt>
21. <stmt>           -> <func_call>
22. <stmt>           -> <block>
23. <var_decl>       -> var IDENTIFIER EOL
24. <assign>         -> IDENTIFIER = <expr>
25. <assign>         -> IDENTIFIER . IDENTIFIER = <expr>
26. <if_stmt>        -> if ( <expr> ) <block> else <block>
27. <while_stmt>     -> while ( <expr> ) <block>
28. <return_stmt>    -> return <expr>
29. <func_call>      -> IDENTIFIER ( <arg_list> )
30. <func_call>      -> Ifj . IDENTIFIER ( <arg_list> )
31. <arg_list>       -> )
32. <arg_list>       -> <expr> <arg_list_tail>
33. <arg_list_tail>  -> )
34. <arg_list_tail>  -> , <expr> <arg_list_tail>
35. <expr>           -> <term> <expr_cont>
36. <expr_cont>      -> eps
37. <expr_cont>      -> <op> <expr>
38. <term>           -> IDENTIFIER
39. <term>           -> INT_LITERAL
40. <term>           -> FLOAT_LITERAL
41. <term>           -> STRING_LITERAL
42. <term>           -> null
43. <term>           -> ( <expr> )
44. <op>             -> +
45. <op>             -> -
46. <op>             -> *
47. <op>             -> /
48. <op>             -> ==
49. <op>             -> !=
50. <op>             -> <
51. <op>             -> <=
52. <op>             -> >
53. <op>             -> >=
54. <op>             -> is
55. <eols>           -> EOF
56. <eols>           -> EOL <eols>
\end{verbatim}

\subsection{LL tabuľka}
Rozkladová tabuľka pre LL(1) analýzu. Bunky obsahujú čísla pravidiel gramatiky, prázdne bunky znamenajú syntaktickú chybu.

\begin{table}[h!]
\centering
\tiny
\setlength{\tabcolsep}{1pt}
\begin{tabular}{|l|c|c|c|c|c|c|c|c|c|c|c|c|c|c|c|c|c|c|c|c|c|c|}
\hline
\textbf{Neterminál} & \textbf{(} & \textbf{)} & \textbf{,} & \textbf{EOF} & \textbf{EOL} & \textbf{FLT} & \textbf{ID} & \textbf{INT} & \textbf{Ifj} & \textbf{Null} & \textbf{/} & \textbf{==} & \textbf{is} & \textbf{+} & \textbf{STR} & \textbf{cls} & \textbf{if} & \textbf{ret} & \textbf{sta} & \textbf{var} & \textbf{whi} & \textbf{\{} \\
\hline
arg\_list        & 32 & 31 &  &  &  & 32 & 32 & 32 &  & 32 &  &  &  &  & 32 &  &  &  &  &  &  &  \\
arg\_list\_tail  &  & 33 & 34 &  &  &  &  &  &  &  &  &  &  &  &  &  &  &  &  &  &  &  \\
assign           &  &  &  &  &  &  & 24 &  &  &  &  &  &  &  &  &  &  &  &  &  &  &  \\
block            &  &  &  &  &  &  &  &  &  &  &  &  &  &  &  &  &  &  &  &  &  & 13 \\
eols             &  &  &  & 55 & 56 &  &  &  &  &  &  &  &  &  &  &  &  &  &  &  &  &  \\
expr             & 35 &  &  &  &  & 35 & 35 & 35 &  & 35 &  &  &  &  & 35 &  &  &  &  &  &  &  \\
expr\_cont       &  & 36 & 36 &  & 36 &  &  &  &  &  & 37 & 37 & 37 & 37 &  &  &  &  &  &  &  &  \\
func             &  &  &  &  &  &  &  &  &  &  &  &  &  &  &  &  &  &  & 6 &  &  &  \\
func\_call       &  &  &  &  &  &  & 29 &  &  &  &  &  &  &  &  &  &  &  &  &  &  &  \\
func\_list       &  &  &  &  &  &  &  &  &  &  &  &  &  &  &  &  &  &  & 5 &  &  & 4 \\
if\_stmt         &  &  &  &  &  &  &  &  &  &  &  &  &  &  &  &  & 26 &  &  &  &  &  \\
main\_func       &  &  &  &  &  &  &  &  &  &  &  &  &  &  &  &  &  &  & 3 &  &  &  \\
op               &  &  &  &  &  &  &  &  &  &  & 47 & 48 & 54 & 44 &  &  &  &  &  &  &  &  \\
param\_list      &  & 9 &  &  &  &  & 10 &  &  &  &  &  &  &  &  &  &  &  &  &  &  &  \\
param\_list\_tail&  & 11 & 12 &  &  &  &  &  &  &  &  &  &  &  &  &  &  &  &  &  &  &  \\
program          &  &  &  &  &  &  &  &  &  &  &  &  &  &  &  & 2 &  &  &  &  &  &  \\
return\_stmt     &  &  &  &  &  &  &  &  &  &  &  &  &  &  &  &  &  & 28 &  &  &  &  \\
stmt             &  &  &  &  &  &  & 17 &  &  &  &  &  &  &  &  &  & 18 & 20 &  & 16 & 19 &  \\
stmt\_list       &  &  &  &  &  &  & 15 &  &  &  &  &  &  &  &  &  & 15 & 15 &  & 15 & 15 & 14 \\
term             & 43 &  &  &  &  & 40 & 38 & 39 &  & 42 &  &  &  &  & 41 &  &  &  &  &  &  &  \\
var\_decl        &  &  &  &  &  &  &  &  &  &  &  &  &  &  &  &  &  &  &  & 23 &  &  \\
while\_stmt      &  &  &  &  &  &  &  &  &  &  &  &  &  &  &  &  &  &  &  &  & 27 &  \\
\hline
\end{tabular}
\caption{LL(1) rozkladová tabuľka (skrátené: ID=IDENTIFIER, FLT=FLOAT\_LIT, INT=INT\_LIT, STR=STRING\_LIT, cls=class, ret=return, sta=static, whi=while)}
\end{table}

Kompletná LL tabuľka obsahuje 24 neterminálov a 35 terminálov. Pre operátory (\texttt{op}) tabuľka obsahuje pravidlá 44--54 pre jednotlivé operátory (+, -, *, /, ==, !=, <, <=, >, >=, is).

\section{Tabuľka symbolov}
Tabuľka symbolov je kľúčová dátová štruktúra, ktorá uchováva informácie o všetkých identifikátoroch (premenných, funkciách) počas prekladu.

\subsection{Implementácia AVL stromom}
Podľa požiadaviek predmetu IAL sme tabuľku symbolov implementovali ako AVL strom v module \texttt{symtable.c}. Použitie samovyvažujúceho binárneho vyhľadávacieho stromu garantuje logaritmickú časovú zložitosť $O(\log n)$ pre operácie vkladania, vyhľadávania aj mazania.

Vyváženosť stromu je zabezpečená rotáciami po každom vložení:
\begin{itemize}
    \item \texttt{far\_left()} -- LL rotácia (pravá rotácia)
    \item \texttt{far\_right()} -- RR rotácia (ľavá rotácia)
    \item \texttt{left\_right()} -- LR rotácia (dvojitá)
    \item \texttt{right\_left()} -- RL rotácia (dvojitá)
\end{itemize}

\subsection{Ukladané informácie}
Každý uzol stromu obsahuje:
\begin{itemize}
    \item \textbf{Kľúč} -- názov symbolu (pre funkcie manglované meno)
    \item \textbf{Typ symbolu} -- premenná lokálna/globálna, funkcia, getter, setter
    \item \textbf{Pre premenné:} \texttt{block\_id}, \texttt{nesting\_level}, príznak inicializácie
    \item \textbf{Pre funkcie:} počet parametrov, typy parametrov, návratový typ
\end{itemize}

\subsection{Správa rozsahov platnosti}
Pre správnu implementáciu rozsahov platnosti (scopes) používame zásobníkový systém blokov. Každý blok (\texttt{if}, \texttt{while}, vnútorné bloky) dostáva unikátne \texttt{block\_id} z počítadla. Lokálne premenné sa interne premenúvajú na tvar \texttt{meno\$bN}, kde N je block ID, čo automaticky rieši shadowing premenných.

Funkcia \texttt{symtable\_search\_var\_scoped()} prehľadáva premenné od aktuálneho bloku smerom k vonkajším blokom, čo zabezpečuje správne nájdenie najvnútornejšej viditeľnej definície.

\section{Sémantická analýza}
Sémantická analýza prebieha integrovane so syntaktickou analýzou a vykonáva nasledujúce kontroly:
\begin{itemize}
    \item Kontrola definície premenných a funkcií pred ich použitím
    \item Kontrola redefinícií v rovnakom rozsahu platnosti
    \item Kontrola počtu parametrov pri volaní funkcií
    \item Typová kontrola literálov vo výrazoch
\end{itemize}

\subsection{Odložená validácia funkcií}
Volania funkcií sa zaznamenávajú do linked listu a validujú sa až po spracovaní celého programu. To umožňuje volať funkcie definované neskôr v kóde.

\subsection{Sledovanie premenných funkcie}
Pre potreby generovania DEFVAR inštrukcií na začiatku funkcie si udržiavame zoznam všetkých premenných použitých vo funkcii v poli \texttt{ifj\_function\_vars}.

\section{Generovanie IFJcode25}
Modul \texttt{generator.c} generuje kód syntaxou riadeným prekladom -- kód sa produkuje priamo počas parsovania.

\subsection{Systém bufferov}
Implementovali sme trojitý buffrovací systém pomocou \texttt{open\_memstream()}:
\begin{enumerate}
    \item \textbf{user\_functions\_buffer} -- zhromažďuje kód všetkých užívateľských funkcií (okrem main)
    \item \textbf{main\_function\_buffer} -- osobitne uchováva telo main funkcie
    \item \textbf{function\_body\_buffer} -- dočasný buffer pre aktuálne spracovávanú funkciu
\end{enumerate}

Tento prístup umožňuje vygenerovať výstup v správnom poradí: najprv hlavička so skokom na main, potom vstavané funkcie, užívateľské funkcie a nakoniec main.

\subsection{Odložené DEFVAR}
IFJcode25 vyžaduje, aby DEFVAR inštrukcie pre lokálne premenné boli na začiatku funkcie. Keďže premenné môžu byť deklarované kdekoľvek v tele funkcie, zbierame ich do zoznamu \texttt{ifj\_function\_vars} a DEFVAR generujeme až po spracovaní celého tela funkcie pomocou \texttt{generate\_all\_function\_defvars()}.

\subsection{Escaping reťazcov}
Funkcia \texttt{convert\_to\_escaped\_string()} konvertuje reťazce do formátu vyžadovaného IFJcode25. Znaky s ASCII hodnotou $\leq 32$, mriežka (\#) a spätné lomítko sa konvertujú na \textbackslash xyz formát (trojciferný ASCII kód).

\subsection{Polymorfné operátory}
Operátory \texttt{+} a \texttt{*} sú polymorfné -- ich správanie závisí od typov operandov. Generujeme runtime kontroly pomocou inštrukcie \texttt{TYPE}, ktoré zisťujú typy operandov na zásobníku a vetvia výpočet na správnu variantu (sčítanie vs. konkatenácia, násobenie vs. opakovanie reťazca).

\subsection{Labely a rámce}
Počítadlo \texttt{ifj\_label\_counter} generuje unikátne návestia pre riadiace štruktúry. Funkcie používajú CREATEFRAME/PUSHFRAME pre vytvorenie lokálneho rámca. Parametre sa vyberajú zo zásobníku v opačnom poradí (posledný vložený = prvý vybraný).

 \clearpage
\section{Chybové hlásenia}
Prekladač vracia nasledujúce návratové kódy chýb:

\begin{table}[h!]
    \centering
    \begin{tabular}{|c|l|}
        \hline
        \textbf{Kód} & \textbf{Popis chyby} \\
        \hline
        1 & Chyba v rámci lexikálnej analýzy (ERR\_LEXICAL) \\
        2 & Chyba v rámci syntaktickej analýzy (ERR\_SYNTAX) \\
        3 & Nedefinovaná funkcia alebo premenná (ERR\_SEM\_UNDEF) \\
        4 & Redefinícia funkcie alebo premennej (ERR\_SEM\_REDEF) \\
        5 & Nesprávny počet parametrov/typ vstavanej f. (ERR\_SEM\_PARAMS) \\
        6 & Chyba typovej kompatibility vo výrazoch (ERR\_SEM\_TYPE\_COMPAT) \\
        10 & Ostatné sémantické chyby (ERR\_SEM\_OTHER) \\
        25 & Behová chyba -- zlý typ parametru (ERR\_RUNTIME\_PARAM\_TYPE) \\
        26 & Behová chyba -- typová kompatibilita (ERR\_RUNTIME\_TYPE\_COMPAT) \\
        99 & Interná chyba prekladača (ERR\_INTERNAL) \\
        \hline
    \end{tabular}
    \caption{Tabuľka návratových kódov}
\end{table}

\section{Testovanie}
Súčasťou projektu sú testovacie súbory \texttt{scanner\_test.c} a \texttt{symtable\_test.c}.
Tieto súbory boli použité na testovanie pre pozitívne testy syntakticky správnych programov a negatívne testy lexikálnych, syntaktických a sémantických chýb.
Samozrejme, toto nebolo jediné testovanie -- testovanie prebiehalo aj mimo týchto súborov počas celej doby práce na projekte.

\section{Práca v tíme}
Projekt bol vyvíjaný v tíme štyroch členov. Na začiatku sme si stanovili rozdelenie úloh podľa vertikálneho prístupu, kde každý člen zodpovedá za určitú časť prekladača.

\subsection{Rozdelenie úloh}
\begin{itemize}
    \item \textbf{Martin Michálik} -- vedenie projektu, generátor kódu, parser, koordinácia
    \item \textbf{Matúš Magyar} -- parser, generator, testovanie, dokumentácia
    \item \textbf{Šimon Škoda} -- lexikálny analyzátor, tabuľka symbolov, precedenčná analýza
    \item \textbf{Jaroslav Vrbiniak} -- tabuľka symbolov, testovanie
\end{itemize}

\subsection{Komunikácia a verzia}
Pre správu zdrojového kódu sme používali verzovací systém Git s repozitárom na GitHub. Pravidelne sme sa stretávali na konzultáciách, kde sme riešili návrh rozhraní medzi modulmi a aktuálne problémy.

\section{Záver}
Projekt implementuje funkcionalitu prekladača IFJ25 podľa špecifikácie. Kód je modulárny a testovaný. Tabuľka symbolov implementuje AVL strom pre efektívne vyhľadávanie.
Kľúčom tohto projektu bolo rozšíriť si naše programovacie schopnosti v jazyku C, naučiť sa pracovať v tíme, pochopiť, ako funguje prekladač a ako ho vytvoriť. Projekt je možné rozšíriť o ďalšie variácie funkcionalít.

\end{document}